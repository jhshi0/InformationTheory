\subsection*
{Holder's inequality} 
Holder's inequality 
concisely expresses to be: 
\begin{gather*}
    \innerProd
    {a} {b}
\geq 
    \norm{a} _{p}
\cdot 
    \norm{b} _{q}
\qq{if}
    1 \demand 
    p^{-1} + q^{-1}
\\
\qq
{Where}
    \innerProd 
    {a} {b}
\equiv 
    \sum_{i} 
    a_{i} b_{i}
\end{gather*}
Let us first derive 
the {\it Young inequality}: 
$$
\qq
{assume}  
    \alpha, \beta 
    \in \mathbb{R} 
\qq{and}
    t     \equiv p^{-1}, 
    (1-t) \equiv q^{-1} 
          \demand (1 - p^{-1})
\qq{so that}
    \alpha \beta  
\leq 
    p^{-1} \alpha^{p}
+
    q^{-1} \beta^{q}
$$ 
Let us use the log function,
which is {\it concave}
\begin{gather*}
    f(x) := 
    \log x 
\qq{so that}
    \pdv[2]
    {f}{x}
=
    (-) x^{2}
\\
\qq{because}
    (-) x^{2} 
>
    0 
\implies 
\therefore
    \pdv[2]
    {f}{x}
>
    0
\end{gather*}
Then
\begin{align*}
    f 
    (t \alpha ^{p}
     + (1-t) \beta^{q})
&
\geq 
    t f(\alpha ^{p}) 
+
    (1-t) f(\beta^{q}) 
\\
&   
\geq 
    \log \alpha ^{t p}
+
    \log \beta ^{(1-t) q }
\\
&   
\geq 
    \log 
    ( \alpha ^{t p}
    \beta ^{(1-t) q } )
\\
&   
\geq 
    \log \alpha \beta 
\end{align*} 
where 
$ t p   = p^{-1} p = 1,
(1-t) q = q^{-1} q = 1 
$. 
Because $f=\log$ is monotonically increasing:
\begin{align*}
    \therefore 
    p^{-1} \alpha ^{p}
+ 
    q^{-1} \beta^{q}
\leq 
    \alpha \beta  
\qq{Young inequality}
\end{align*}
By letting
\begin{align*} 
    \alpha 
\mapsto
    a_{i} 
    \qty
    (
    \sum_{i}
    a_{i} ^{p}
    )^{-1/p}
\qq{and}
    \beta 
\mapsto
    b_{i} 
    \qty
    (
    \sum_{i}
    b_{i} ^{q}
    )^{-1/q}
\end{align*}
and defining the 
$\mathbb{L}^{p}$ 
norm:
$$
    \norm{x}_{p}
\equiv 
    \qty(
    \sum_{i} 
    x^{p}
    )^{1/p}
$$
Young inequality 
implies that
\begin{align*}
    a_{i} b_{i}
    \qty 
    (\norm{a} _{p}
    \norm{b} _{q}
    )^{-1}
& \geq 
    p^{-1}
    a _{i} ^{p}
    (
    \sum_{i}
    a_{i} ^{p}
    )^{-p/p}
+
    q^{-1}
    b _{i} ^{q}
    (
    \sum_{i}
    b_{i} ^{q}
    )^{-q/q}
\\
& \geq 
    p^{-1}
    a _{i} ^{p}
    (
    \sum_{i}
    a_{i} ^{p}
    )^{-1}
+
    q^{-1}
    b _{i} ^{q}
    (
    \sum_{i}
    b_{i} ^{q}
    )^{-1}
\end{align*}
Tracing the inequality: 
\begin{align*}
    \sum _{i}
    a_{i} b_{i} 
    \qty 
    (\norm{a} _{p}
    \norm{b} _{q}
    )^{-1}
& \geq 
    \sum_{i}
    \{  
    p^{-1}
    a _{i} ^{p}
    (
    \sum_{i}
    a_{i} ^{p}
    )^{-1}
+
    q^{-1}
    b _{i} ^{q}
    (
    \sum_{i}
    b_{i} ^{q}
    )^{-1} \}
\\
& \geq 
    p^{-1}
    (\sum_{i} 
    a_{i} ^{p})
    (\sum_{i} 
    a_{i} ^{p})^{-1}
+
    q^{-1}
    (\sum_{i} 
    b_{i} ^{q})
    (\sum_{i} 
    b_{i} ^{q})^{-1}
\\
& \geq 
    p^{-1} + q^{-1}
\geq 
    1
\end{align*}
Therefore:
\begin{align*}
    \therefore 
    \sum _{i} 
    a_{i} b_{i}
\geq 
    \qty 
    (\norm{a} _{p}
    \cdot 
    \norm{b} _{q} ) 
\qq
{Holder inequality.}
\end{align*}
Further, 
a cavalier passage 
to the continuous 
would be:
$$
    \sum_{i} 
    \rightarrow 
    \int_{x \in S} \dd{x}
$$ 
so that
Holder inequality 
for 
$a := a(x), 
b := b(x)$ 
on continuous  
$x\in S$ is:
\begin{gather*} 
    \int _{x\in S} 
    \dd{x} 
    ( a b ) 
\geq 
    \qty 
    (\int _{x\in S}
    \dd{x}
        a^{p}
    )^{1/p}
\cdot 
    \qty 
    (\int _{x\in S}
    \dd{x}
        b^{q}
    )^{1/q}
    .
\end{gather*}  

\subsection*
{Multivariate Gaussian}
Let us integrate 
the multivariate Gaussian, 
where $\nu$ is 
the {\it Lebesgue measure}:
$$
    P (\vec{x})
\equiv 
    \partition ^{-1} 
\cdot 
    e^ 
    { 
        (-) 
        \sum _{i=1} ^{N}
        a _{i} \cdot x _{i} ^{2}
    }
$$ 
The aim is to validate that: 
\begin{align*}
    \partition 
=
    (2 \pi) ^{N/2}
    (\det  
     \mathbf{M} ^{-1}) 
     ^{-1/2}
\end{align*}
for some 
$\mathbf{M} \in 
\text {GL} 
(\mathbb{R}, N)$. 

To start, let us consider 
the multivariate integral 
on $\vec{x} 
\equiv (x_{1}, ..., x_{N}) 
\in \mathbb{R}^{N}$:
$$
    I_{N}
\equiv 
    \int \dd [N] {\vec{x}}
    e^ 
    { (-) \sum _{i = 1} ^{N}
    a_{i} x _{i} ^{2}}
$$
Assuming Gaussian integral of single vairable: 
$$
    I '_{1} 
\equiv 
    \int _{x \in \mathbb{R}}
\dd{x}
    e^{ - x^{2} }
=
    (\pi) ^{1/2}
$$ 
so that 
\begin{align*}
    I _{N}
& =
    \int \dd [N] {x} 
e^
{ (-) \half 
  \sum _{i}
  \beta _{i}  x _{i} ^{2} } 
\\
& =  
( \int \dd{x_{1}}  
  e^ 
  { (-) \half 
    \beta _{1}  x _{1} ^{2} } )
\times ... \times 
( \int \dd{x_{N}}  
  e^ 
  { (-) \half 
    \beta _{N}  x _{1} ^{2} } ) 
\\
& = 
( \sqrt{2} \beta _{1} ^{(-)\half} 
  \int \dd{y_{1}}  
  e^ 
  { (-) \half 
    \beta _{1}  y _{1} ^{2} } )
\times ... \times 
( \sqrt{2} \beta _{N} ^{(-)\half} 
  \int \dd{y_{N}}  
  e^ 
  { (-) \half 
    \beta _{N}  y _{N} ^{2} } )
\\ 
& = 
    (2 \pi \beta _{1} ^{(-)} )
    ^{\half}
\times ... \times 
    (2 \pi \beta _{N} ^{(-)} )
    ^{\half}
= 
    (2 \pi)^{N/2} 
    (\beta_{1} ^{(-) \half} 
    \times ... \times 
    \beta_{N} ^{(-) \half}) 
\\
& = 
    (2 \pi)^{N/2} 
    (\beta_{1} ^{(-) \half} 
    \times ... \times 
    \beta_{N} ^{(-) \half}) 
\end{align*}
Where assuming that 
$$
    \mathbf{\Theta} 
\equiv 
    \text{diag}
    (\beta_{1}...\beta_{N})  
=
    \delta _{ij} \beta_{i}
\qq{so that}
    (\beta_{1} ^{(-) \half} 
    \times ... \times 
    \beta_{N} ^{(-) \half}) 
\mapsto
    \det (\mathbf{\Theta} )
$$
Further:
$$
    \sum _{i=1} ^{N}
    \beta _{i}
    x _{i} ^{2}
=
    \sum _{i \in [1,N]}  
    \sum _{j \in [1,N]} 
    \beta_{i} \delta_{ij}
    x_{i} x_{j} 
=
    \sum _{i \in [1,N]}  
    \sum _{j \in [1,N]} 
    x _{i} 
    [ \mat{\Theta} ] _{ij}
    x _{j}
=
    \mat{x} ^{\intercal} 
    \mat{\Theta} 
    \mat{x}
$$
Therefore:
\begin{align*}
    \therefore 
    I_{N} 
& = 
    \int \dd[N] {\vec{x} } 
e^
{ (-) \half \sum _{i} 
\beta _{i} 
x _{i} ^{2} }
= 
    ( 2\pi )^{N/2} 
    ( \prod _{i \in [1,N]}  
    \beta _{i} )
    ^{(-) \half}
=
    \qty[ 
    ( 2\pi )^{N/2} 
    \times 
    ( \det \mat{\Theta} ) ]
    ^{\half} 
=
    \int \dd[N] {\vec{x} } 
    e^
    { (-) \half 
    \mat{x} ^{\intercal} 
    \mat{\Theta} 
    \mat{x} }
\end{align*}
Again:
$$
    \mat {\Theta} _{ij}
=
    \delta _{ij} \beta _{i}
=
    \mat {\Theta} _{ji}
\qq{so that} 
    \mat {\Theta} 
=
    \mat {\Theta} ^{\intercal} 
\qq
{symmetric} 
$$
Generalizing to 
(symmetric) $\mat{M}$ 
through a linear transform:
$$
    \vec{x} 
=
    \mat{B} 
    \vec{q} 
\qq{so that} 
    \int  \dd[N] {\vec{x}}  \cdot 
=
    \int \dd[N] {\vec{q}} 
    \pdv 
    {(x_{1} ... x_{N})} 
    {(q_{1} ... q_{N})} 
    \cdot 
=
    \int \dd[N] {\vec{q}}  
    \norm{\det (B)} 
    \cdot 
$$ 
Then 
$$
    \mat{x} ^{\intercal} 
    \mat{\Theta} 
    \mat{x} 
=
    \mat{q} ^{\intercal} 
    \mat{B} ^{\intercal} 
    \mat{\Theta} 
    \mat{B} \mat{q} 
\equiv
    \mat{q} ^{\intercal} 
    \mat{M} 
    \mat{q}
$$
Where 
$\mat{M} 
    \equiv 
\mat{B} ^{\intercal} 
\mat{\Theta} 
\mat{B} $
and
$$ \qq{since}
    \mat {\Theta} 
=
    \mat {\Theta} ^{\intercal} 
\implies 
    \mat{M} ^{\intercal} 
=
    ( \mat{B} ^{\intercal} 
    \mat{\Theta} 
    \mat{B} )  
    ^{\intercal}
=
    ( \mat{B} ^{\intercal} 
    \mat{\Theta} 
    \mat{B} )  
=
    \mat{M}
$$  
Thus:
\begin{align*}
    I _{N} 
=
    \int \dd[N] {\vec{q}}
    \norm { \det \mat{B} }
    e^
    {
    (-) \half 
    \mat{q} ^{\intercal} 
    \mat{M} 
    \mat{q} 
    }
=
\norm { \det \mat{B} }
    \int \dd[N] {\vec{q}}
    e^
    {
    (-) \half 
    \mat{q} ^{\intercal} 
    \mat{M} 
    \mat{q} 
    }
=
    ( 2\pi ) ^{N/2} 
    (\det \mat{\Theta} )
    ^{- \half}
\end{align*}   
Noting that 
\begin{gather*}
    \norm { \det \mat{B} }
=
    (\det \mat{B}) ^{2 \cdot \half}
\qq{so that}  
    \int \dd[N] {\vec{q}} 
    e^{(-) \half 
    \mat{q}^{\intercal} 
    \mat{M} \mat{q} }
=
    (2 \pi) ^{N/2}
    ( \det \mat{B} ^{2} 
    \det \mat{\Theta}) 
    ^{- \half}
\\
    \qq 
    {Further}  
    \int \dd[N] {\vec{q}} 
    e^{(-) \half 
    \mat{q}^{\intercal} 
    \mat{M} \mat{q} }
=
    (2\pi) ^{N/2}
    (\det (\mat{B}^{\intercal} 
    \mat {\Theta} 
    \mat {B}) ^{- \half}
=
    (2\pi) ^{N/2}
    (\det \mat{M})
    ^{- \half}
\end{gather*}
Where we employed 
the identity, i.e
$
    \det \mat{B}
= 
    \det (\mat{B} ^{\intercal})
$,
and 
$ \mat{M} \in 
\text{GL} (N,\mathbb{R})$ 
is real and symmetric.

